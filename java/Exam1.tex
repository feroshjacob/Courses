\documentclass[12pt]{exam}
\usepackage[utf8]{inputenc}

\usepackage[margin=1in]{geometry}
\usepackage{amsmath,amssymb}
\usepackage{multicol}
\usepackage{listings}
\newcommand{\class}{Programming Principles I}
\newcommand{\term}{Fall 2017}
\newcommand{\examnum}{Exam 1}
\newcommand{\examdate}{09/12/2017}
\newcommand{\timelimit}{120 Minutes}
\usepackage{color}

\definecolor{pblue}{rgb}{0.13,0.13,1}
\definecolor{pgreen}{rgb}{0,0.5,0}
\definecolor{pred}{rgb}{0.9,0,0}
\definecolor{pgrey}{rgb}{0.46,0.45,0.48}
\pagestyle{head}
\firstpageheader{}{}{}
\runningheader{\class}{\examnum\ - Page \thepage\ of \numpages}{\examdate}
\runningheadrule
\lstset{language=Java,
  showspaces=false,
  showtabs=false,
  breaklines=true,
  showstringspaces=false,
  breakatwhitespace=true,
  commentstyle=\color{pgreen},
  keywordstyle=\color{pblue},
  stringstyle=\color{pred},
  basicstyle=\ttfamily,
  moredelim=[il][\textcolor{pgrey}]{$$},
  moredelim=[is][\textcolor{pgrey}]{\%\%}{\%\%}
}

\begin{document}

\noindent
\begin{tabular*}{\textwidth}{l @{\extracolsep{\fill}} r @{\extracolsep{6pt}} l}
\textbf{\class} & \textbf{Name:} & \makebox[2in]{\hrulefill}\\
\textbf{\term} &&\\
\textbf{\examnum} &&\\
\textbf{\examdate} &&\\
\textbf{Time Limit: \timelimit} & Instructor: \textbf{Dr. Ferosh Jacob} & 
\end{tabular*}\\
\rule[2ex]{\textwidth}{2pt}
\begin{itemize}
\item Please read the questions very carefully.
    \item This exam contains \numpages\ pages (including this cover page) and \numquestions\ questions with total  points  \numpoints. 
\item Try to attempt as many as possible. However, the maximum score you can reach is 100.
\item Good luck!
\end{itemize}

\begin{center}
Grade Table (for teacher use only)\\
\addpoints
\gradetable[v][questions]
\end{center}

\noindent
\rule[2ex]{\textwidth}{2pt}

\begin{questions}

\question[5]  Explain the concept of computer program from Chapter 1.
\makeemptybox{2in}
\addpoints

\question[5] Name and explain the three error types and give one example of each type you may encounter when writing a Java program
\makeemptybox{2in}
\addpoints

\question[5] Explain the ``Reserved Words'' in Java.

\makeemptybox{2in}
\addpoints

%\newpage
\question[5] Explain the term ``High-Level Languages.''

\makeemptybox{2in}
\addpoints


\question[5] Explain the term ``Operating System.''

\makeemptybox{2in}
\addpoints


\question[5] List three editions of Java Development Kit (JDK)

\begin{enumerate}
    \item 
    \item
    \item
\end{enumerate}
\addpoints

\question[5] Explain the terms ``Syntax rules'' and ``Semantics''

\makeemptybox{2in}
\addpoints

\question[5] List the eight Java primitive data types

\begin{enumerate}
    \item 
    \item
    \item
    \item
    \item
    \item
    \item 
    \item
\end{enumerate}
\addpoints

\question[5]. Explain the difference between \texttt{float} type and \texttt{double} type.

\makeemptybox{2in}
\addpoints

\question[5]. Show in details how Java would evaluate the following expression and show the order of operations:

$X = 100 – (40 + (200 / (10 - 5) * 2) / 4) + 50$

\makeemptybox{2in}
\addpoints
%\newpage
\question . Write Java code (correct syntax) for each of the following requirements. Do not write complete programs, just what is required for each part. 

\begin{parts}
\part[4] Declare two integer variables (called A and B); initialize A to 20 and B to 40; assign the sum of A and B to K.
\makeemptybox{2in}

\part[4] Declare two integer variables (called V and W); initialize V to 50;  to assign $V^2$ to W;  then  printout the square of V with this label:


\texttt{ The Square  of N = 2500 }

\makeemptybox{2in}
\part[4] Declare a variable of type double (call it \texttt{input}) and initialize it to $123.45$; then change its value by dividing it by $5$; finally assign the value of  variable $input$ to another integer variable (call it $int\_value$) using explicit type casting. Make sure all variables are declared.


\makeemptybox{2in}
\part[4] Declare a boolean variable (call it \texttt{flag}) and initialize it to {false}; then assign the expression $‘A’ < ‘a’$  to variable $flag$; and finally print out the value of  variable flag with this label   

\texttt{Flag =}
\makeemptybox{2in}

\part[4]
Assume you have a scanner object (called \texttt{input}). Declare an integer variable (call it \texttt{value}); initialize the variable with an input from the user; check if the variable (\texttt{value})is even or not; finally, print proper message such as ``value is even'' or ``value is odd''.
\makeemptybox{2in}

%\newpage
\part[4] Create a scanner object (called \texttt{scan}). Declare a string variable (call it \texttt{name}); initialize the variable with an input from the user; finally, display the value of variable \texttt{name} on the screen with proper label such as

\texttt{The name is Edward.}
\makeemptybox{2in}
\end{parts}
\addpoints

\question  Evaluate each of the following Java code write the exact output. Show all spaces and tabs properly in the outputs.
\begin{parts}
\part[2]  Code:
\begin{lstlisting}
boolean flag = false;
boolean output = flag ^ true;
System.out.println("Output = " + output);
\end{lstlisting}
\makeemptybox{0.5in}


\part[2]  Code:
\begin{lstlisting}
System.out.println("Your points is \n\t" + 21 + 21);  
\end{lstlisting}
\makeemptybox{0.5in}

\part[2]  Code:
\begin{lstlisting}
double a = (int) (50.0 / 10.0);
System.out.println("Answer a = " + a);\end{lstlisting}
\makeemptybox{0.5in}

%\newpage
\part[2]  Code:
\begin{lstlisting}
System.out.println("Output = \t\t" + (10+20));
\end{lstlisting}
\makeemptybox{0.5in}


\part[2]  Code:
\begin{lstlisting}
int b = 28/2%4;
System.out.println("Answer is b = " + b);
\end{lstlisting}
\makeemptybox{0.5in}
\end{parts}
\addpoints

\question  Write boolean expressions. 
\begin{parts}
\part[2] Write a boolean expression that evaluates to true if \texttt{age} is greater than 15 and less than 21.
\makeemptybox{0.5in}
\part[2] Write a boolean expression that evaluates to true if \texttt{weight} is greater than or equal to 50 pounds or \texttt{height} is greater than 60 inches.
\makeemptybox{0.5in}
\part[2] Write a boolean expression that evaluates to true if \texttt{speed} is greater than or equal to 80 miles and  \texttt{distance} is less than or equal to 100 miles.
\makeemptybox{0.5in}
\end{parts}
\addpoints

\question What is the output of the following code segment?
\begin{parts}
\part[5] Code:
\begin{lstlisting}
int N = 30;
if (N % 2 == 0)
    System.out.println(N + " is Even");
if (N % 5 == 0)
    System.out.println(N + " is multiple of 5");
if (N % 2 == 0)
    System.out.println(N + " is Even");
else if (N % 5 == 0)
    System.out.println(N + " is multiple of 5");
\end{lstlisting}
\makeemptybox{1in}

\part[5] Code:
\begin{lstlisting}
int x=2, y=3, z=4;
System.out.println("(x<y && y<z) is " + (x<y && y<z));
System.out.println("(x<y || y<z) is " + (x<y || y<z));
System.out.println("!(x < y) is " + !(x < y));
System.out.println("(x + y < z) is " + (x + y < z));
System.out.println("(x + y > z) is " + (x + y > z));  
\end{lstlisting}
\makeemptybox{1in}

\part[5] Trace and print out the exact output of the following Java code.

\begin{lstlisting}
boolean flag1 = true;
boolean flag2 = true;
int count = 0; 

if ((flag1 && false) || (!flag2))
    count = count + 10;
if ((flag1 || false) && (flag1))
    count = count + 30;
if (flag1 && flag2 && true)
    count = count + 20;
if ((!flag1 || true) && false)
    count = count + 10;
if ((true && flag2) || (!flag1)
    count = count + 10;

System.out.println("The variable count is " + count);
\end{lstlisting}
\makeemptybox{0.5in}

\part[5] Trace and print out the exact output of the following Java code.

\begin{lstlisting}
import java.util.Scanner;
public class T1_Q11
{
   public static void main (String[] args)
   {
      int limit = 100, num1 = 15, num2 = 40;
      if (limit > num1)
         System.out.println("lemon");
      else
         System.out.println("grape");
      System.out.println("lime");
      System.out.println("lime");
      if (limit <= limit)
      {
         if (num1 == num2)
         {
            System.out.println("lemon juice");
         }
         System.out.println("lime juice");
      }
      else System.out.println("grape");
      System.out.println("lemon"); 
\end{lstlisting}
\makeemptybox{1in}
\end{parts}
\addpoints
\question[10] Complete the following Java program to perform the tasks indicated in the in-line comments below.
\begin{lstlisting}
import java.util.Scanner;
public class T1_Q12
{
  public static void main (String[] args)
  {
     int quantity;
     double unitPrice, totalCost;
     Scanner scan = new Scanner (System.in);
     
     //prompt the user to enter quantity and read it into the proper variable 
     
     

     
     //prompt the user to enter unit price and read it into the proper variable




     //Compute the total cost



     // printout all variables with proper labels and lineup the outputs using tabs
    
    
    
    
    
    }
}

\end{lstlisting}
\addpoints
%\question[2] One of these things is not like the others; one of these
%things is not the same. Which one is different?
%\begin{choices}
%\choice John
%choice Paul
%\choice George
%\choice Ringo
%\choice Socrates
%\end{choices}

%\question[2] One of these things is not like the others; one of these
%things is not the same. Which one is different?
%\begin{oneparchoices}
%\choice John
%\choice Paul
%\choice George
%\choice Ringo
%\choice Socrates
%\end{oneparchoices}

%\question[3] Mark box if true.
%\addpoints
%\begin{checkboxes}
%\choice 2+2=4
%\choice $\frac{d}{dx} (x^2+1) = 2x+1$
%\choice The Moon is made of cheese.
%\end{checkboxes}

%{%
%\checkboxchar{$\Box$} % changing checkbox style locally
%\question[3] Mark box if true.
%\addpoints
%\begin{checkboxes}
%\choice 2+2=4
%\choice $\frac{d}{dx} (x^2+1) = 2x+1$
%\choice The Moon is made of cheese.
%\end{checkboxes}
%}%

%{%
%% changing choice items style locally
%\renewcommand*\thechoice{\arabic{choice}} 
%\renewcommand*\choicelabel{\thechoice)}
%
%\question[2] Element with $Z=92$ is:
%\begin{multicols}{2}
%\begin{choices}
%\choice H
%\choice O
%\choice F
%\choice S
%\choice Ba
%\choice Pb
%\choice U
%\choice Pu
%\end{choices}
%\end{multicols}
%}%

%\question[10]
%In no more than one paragraph, explain why the earth is round.
%\makeemptybox{2in}

%\question[20]
%Explain blah, blah\ldots
%\makeemptybox{\fill}

%\newpage

%\question[20]
%Explain blah, blah\ldots
%\fillwithlines{\fill}

%\newpage

%\question[20]
%Explain blah, blah\ldots
%\fillwithdottedlines{8em}

\end{questions}

\end{document}

