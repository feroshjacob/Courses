\documentclass[12pt]{exam}
\usepackage[utf8]{inputenc}

\usepackage[margin=1in]{geometry}
\usepackage{amsmath,amssymb}
\usepackage{multicol}
\usepackage{listings}
\newcommand{\class}{Programming Principles I}
\newcommand{\term}{Fall 2017}
\newcommand{\examnum}{Exam 1}
\newcommand{\examdate}{09/12/2017}
\newcommand{\timelimit}{120 Minutes}
\usepackage{color}

\definecolor{pblue}{rgb}{0.13,0.13,1}
\definecolor{pgreen}{rgb}{0,0.5,0}
\definecolor{pred}{rgb}{0.9,0,0}
\definecolor{pgrey}{rgb}{0.46,0.45,0.48}
\pagestyle{head}
\firstpageheader{}{}{}
\runningheader{\class}{\examnum\ - Page \thepage\ of \numpages}{\examdate}
\runningheadrule
\lstset{language=Java,
  showspaces=false,
  showtabs=false,
  breaklines=true,
  showstringspaces=false,
  breakatwhitespace=true,
  commentstyle=\color{pgreen},
  keywordstyle=\color{pblue},
  stringstyle=\color{pred},
  basicstyle=\ttfamily,
  moredelim=[il][\textcolor{pgrey}]{$$},
  moredelim=[is][\textcolor{pgrey}]{\%\%}{\%\%}
}

\begin{document}

\noindent
\begin{tabular*}{\textwidth}{l @{\extracolsep{\fill}} r @{\extracolsep{6pt}} l}
\textbf{\class} & \textbf{Name:} & \makebox[2in]{\hrulefill}\\
\textbf{\term} &&\\
\textbf{\examnum} &&\\
\textbf{\examdate} &&\\
\textbf{Time Limit: \timelimit} & Instructor: \textbf{Dr. Ferosh Jacob} & 
\end{tabular*}\\
\rule[2ex]{\textwidth}{2pt}
\begin{itemize}
\item Please read the questions very carefully.
    \item This exam contains \numpages\ pages (including this cover page) and \numquestions\ questions with total  points  \numpoints. 
\item Try to attempt as many as possible. However, the maximum score you can reach is 100.
\item Good luck!
\end{itemize}

\begin{center}
Grade Table (for teacher use only)\\
\addpoints
\gradetable[v][questions]
\end{center}

\noindent
\rule[2ex]{\textwidth}{2pt}
\pagebreak
\begin{questions}

\section{Trace a code segment}
\question[10] What is the exact output of the following code segment? 


 Code:
\begin{lstlisting}
      for (int i=1;  i<=4;  i++)
      {
         for (int j=1; j<=4; j++)
             System.out.print((j-i) + "\t");
         System.out.println();
      } 
\end{lstlisting}
\makeemptybox{2in}
\addpoints


\question[10] What is the exact output of the following code segment? 


 Code:
\begin{lstlisting}
       for (int i=0; i<3; i=i+1)
       {
           for (int j=i*3; j<i*3+3; j=j+1)
               System.out.print(i + ";  ");
           System.out.println();
       }
\end{lstlisting}
\makeemptybox{2in}
\addpoints

\question[10] What is the exact output of the following code segment? 


 Code:
\begin{lstlisting}
      int counter; 
      for (counter = 7; counter <= 16; counter = counter + 1)
      { 
         switch ((counter % 10))
         {
            case 0: System.out.print(", "); 
               break;
            case 1: System.out.print("OFTEN "); 
               break;
            case 2: System.out.print("NEVER "); 
               break; 
            case 3: System.out.print("DONE "); 
               break;
            case 4: System.out.print("VERY "); 
               break;
            case 5: System.out.print("WELL "); 
               break;
            case 6: System.out.print("."); 
               break;
            case 7: System.out.print("WHAT "); 
               break;
            case 8: System.out.print("IS "); 
               break;
            case 9: System.out.print("DONE"); 
               break;
            default: System.out.print("Bad Number. ");
         } 
      }
      System.out.println();

\end{lstlisting}
\makeemptybox{0.5in}
\addpoints

\end{questions}

\end{document}

